Outline of the dissertation 


(Finished)    
1. Introduction 
    1.1 Telescope Signal
        1.1.1 Prediction Photon Count
        1.1.2 Photon Noise
        1.1.3 Diffraction PSF
        1.1.4 Aberrations
   1.2 Atmospheric Turbulence
   1.3 Direct Imaging

(Finished)        
2. Operational theory
    2.1 Intro to Adaptive Optics
        2.1.1 Strehl Ratio
    2.3 Pyramid Wavefront Sensor 
    2.3 Diffraction Theory
        2.3.1 Derivation of Expected Signal from PWFS
        2.3.2 Verification of Diffraction Theory
    2.4 Slopes Maps 
        2.4.1 Derivation
        2.4.2 Verification on LOOPS


(Already written SPIE Conference Proceedings+ additional information) 
3. Design of the MagAO-X WFS
    3.1 Introduction to Optical design (paraxial geometric optics and Seidel Equations)
        3.1.1 Spherical Aberration (and how to reduce it in lens design)
        3.1.2 Chromatic Aberrations (doubles and triplet lenses)
    3.2 MagAO-X
        3.2.1 System Design (design requirements)
        3.2.2 Pyramid Design
        3.2.3 Achromatic Triplet Design
        3.2.4 As built Triplet Results
    3.3 Alignment
        4.3.1 Explanation of considerations in optical alignment 
        4.3.2 MagAO-X PWFS Alignment
        4.3.3 MagAO-X PWFS Current Status

(Written but not integrated)
4. OOMAO Simulations
    4.1 Overview of OOMAO
    4.2 Experiment Details
    4.3 Experimental Results
 
(Major To Do)   
5. The CACTI Testbed
    5.1 Purpose (experimental goals)
    5.2 Design
        5.2.1 Overview on how to design an OAP in zemax
        5.2.2 System Design
            5.2.2.1 Conjugate Imaging
    5.3 Alignment
        5.3.1 DOF
        5.3.2 Mounting/clocking
        5.3.3 Defining Beam lines
        5.3.4 Collimation & Shear plate
        5.3.5 Focal plane alignment (PSF watching + phasics)
        5.3.6 PSF clean up (the Eye Doctor Script)
    5.4 Closing The Loop + brief CACAO summary
    5.5 Experimental Details
    5.6 Strehl Ratio
        6.6.1 Explanation of the Strehl Ratio
        6.6.2 Strehl Class Code
    5.7 Experimental Results
    
6. Conclusions and Future Work
    
        

