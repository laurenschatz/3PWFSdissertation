Known exoplanets are 10 thousand to 10 billion times fainter than their host stars. These high contrast ratios present significant challenges for the design of instruments to image exoplanets directly. Ground-based observatories face the additional challenge of atmospheric turbulence degrading the image quality. Astronomers have developed extreme adaptive optics (ExAO)  instruments that are optimized for correcting atmospheric turbulence and suppressing starlight. Within the next decade, the world will see a new generation of telescopes with diameters up to 39 meters, called the Giant Segmented Mirror Telescopes (GSMT). The GSMTs have the angular resolution and light-collecting power to detect and characterize potentially habitable terrestrial exoplanets for the first time. This will only be achievable if the performance of GSMT-ExAO systems is optimized. Alternative architectures of wavefront sensors are under consideration for GSMT-ExAO instruments considering the trade-offs between detector size, speed, and noise that determine the performance of GSMT-ExAO wavefront control. 

This dissertation aims to develop the three-sided pyramid wavefront sensor (3PWFS) as an alternative GSMT-ExAO wavefront sensor. The 3PWFS  uses fewer detector pixels so it is less sensitive to read noise than the 4PWFS. In this work, I develop a mathematical formalism based on the diffraction theory description of the Foucault knife-edge test that predicts the intensity pattern after the pyramid wavefront sensors (PWFS). I use these results to motivate methods for processing the signals from a 3PWFS and implement this methodology in an end-to-end adaptive optics simulation to compare the performance of the 3PWFS to the 4PWFS. I will describe the design and alignment of the Comprehensive Adaptive Optics and Coronagraph Test Instrument (CACTI), a new ExAO testbed designed with the flexibility to support visiting instruments and to be easily re-configurable to perform multiple experiments. Both a 3PWFS and 4PWFS were integrated into CACTI to demonstrate the operation of a 3PWFS and compare it to the 4PWFS.


